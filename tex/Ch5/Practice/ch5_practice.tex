\documentclass[12pt]{article}

%%%%%%%%%%%%%%%%%%%%%%%%%%%%
%%%%%%%%%%%%%%%%%%%%%%%%%%%%
% Load in packages
\usepackage{amsmath}
\usepackage{amssymb}
\usepackage{hyperref}
\usepackage{graphicx}
\usepackage[top=1in, bottom=1in, left=1in, right=1in]{geometry}

%%%%%%%%%%%%%%%%%%%%%%%%%%%%
%%%%%%%%%%%%%%%%%%%%%%%%%%%%

\begin{document}

\begin{center}
\Large Chapter 5 Practice Problems Solutions

\medskip

\normalsize Elements of Microeconomics

\medskip

\small Discussion section 4
\end{center}

\medskip

\section*{Question 1}
\subsection*{Part A}
Based on your intuition, choose 3 goods for which you think:
\begin{enumerate}
    \item Demand is inelastic
    \item Demand is elastic
    \item Supply is inelastic
    \item Supply is elastic
\end{enumerate}
It might be helpful to add a bit of justification. Does it matter what \textit{time frame} you're thinking about? Does the \textit{scope of the market} matter? Any other factors?

\vspace{5mm}

\subsection*{Part B}
Think about the market for Ford F150s. Do you expect demand to be elastic or inelastic? What about supply? Does this depend on any qualifiers about the time frame and the scope of the market?

\vspace{5mm}

\section*{Question 2}
\subsection*{Part A}
Take two points on a demand curve:
\begin{itemize}
    \item $P_A=12$ and $Q_A=60$
    \item $P_B=8$ and $Q_B=80$
\end{itemize}

Moving from A to B, what is the price elasticity of demand? Show each step clearly.

\vspace{2mm}

Moving from B to A, what is the price elasticity of demand? Again, show each step.

\vspace{5mm}

\subsection*{Part B}
Using the mid-point formula, answer the following questions:
\begin{enumerate}
    \item What is the new base price?
    \item What is the new base quantity?
    \item What is the \% change for quantity?
    \item What is the \% change for price?
    \item What is the price elasticity of demand? Does it matter which point we treat as the start?
\end{enumerate}

\section*{Question 3}
Draw example demand curves which are:
\begin{itemize}
 \item Elastic
 \item Inelastic
 \item Unit elastic
 \item Perfectly elastic
 \item Perfectly inelastic
\end{itemize}
and provide the intuition behind the shape of each.

\vspace{2mm}

\section*{Question 4}
Say price for some good doubles from $P_A$ to $P_B = 2*P_A$. How does total revenue change when:
    \begin{itemize}
        \item Demand is elastic: quantity decreases by 75\%
        \item Demand is inelastic: quantity decreases by 25\%
        \item Demand is unit elastic
    \end{itemize}

\section*{Question 5}
Say we have a linear demand curve:
\begin{itemize}
    \item Quantity demanded is 0 when price is 100
    \item Quantity demanded is 10 when price is 20
\end{itemize}

\medskip

\begin{enumerate}
    \item Calculate the formula for the demand curve (slope and intercept) and draw graphically
    \item Is the elasticity constant? Why or why not?
    \item Pick a few example points, and use the midpoint formula to check the elasticity when:
        \begin{enumerate}
            \item Price is close to 20
            \item Price is close to 0
            \item Price is around 8
        \end{enumerate}
    \item How will total revenue vary as price moves from 0 to 100?
\end{enumerate}

\section*{Question 6}
Let's think about the market for hotel rooms, where we have some people searching for rooms for business travel, others for vacation, and some firms providing hotel rooms:

\begin{center}
    \begin{table}
    \begin{tabular}{|c|c|c|c|}
      \hline
      \textbf{Price} & \textbf{$Q_D$ (Business)} & \textbf{$Q_D$ (Vacation)} & \textbf{$Q_S$ (Firms)} \\
      \hline
      \$150 & 2,100 & 1,000 & 2,300 \\
      \hline
      \$200 & 2,000 & 800   & 2,400 \\
      \hline
      \$250 & 1,900 & 600   & 2,500 \\
      \hline
      \$300 & 1,800 & 400   & 2,600 \\
      \hline
    \end{tabular}
    \caption{Market for airline tickets}
  \end{table}
\end{center}

Which group do you expect to be elastic? Inelastic? Why?

\vspace{2mm}

Calculate the elasticities, and say when the market is inelastic. If you are comfortable with the arithmetic, you may just want to do these calculations in an excel spreadsheet and focus on the intuition.

\end{document}