
% Inbuilt themes in beamer
\documentclass[aspectratio=169]{beamer}
\usepackage{graphicx}
% Theme choice:
\usetheme{CambridgeUS}

% Title page details: 
\title{Chapter 3: Interdependence and Gains from Trade} 
\author{John Green \\
Discussion Section 4}
\date{September 8, 2023}


\begin{document}

% Title page
\begin{frame}
    \titlepage 
\end{frame}

% Outline frame
\begin{frame}{Outline}
    These slides will cover chapter 3, "Interdependence and the Gains from Trade."

    \medskip

    The main takeaways from this chapter are on \textit{advantage} --- both absolute and comparative --- and the benefits of specialization and trade.
\end{frame}

\begin{frame}{Thinking at the margin}
    First, something we glossed over in last week's discussion:

    \begin{center}
        \textit{Economists think at the margin.}
    \end{center}

    \medskip

    More importantly, we believe that profit-maximizing firms and utility-maximizing individuals do the same.

    \medskip

    This means when evaluating a decision, we think about what a small change in behavior will do to an outcome.

\end{frame}

% How people make decisions
\begin{frame}{Absolute advantage}
    \textit{Absolute advantage} means the ability to produce more of a good given a fixed quantity of inputs.
    
    \medskip

    Today, let's work with two restaurants, Stu's Steakhouse and Sandie's Salads. Both of them can produce two dishes: salads and steaks. Given 1000 minutes of labor time, they can produce the following amounts of each dish:

    \begin{table}[h]
  \centering
  \begin{tabular}{|c|c|c|}
    \hline
    \textbf{Restaurant} & \textbf{Steaks} & \textbf{Salads} \\
    \hline
    \textbf{Stu's Steakhouse} & 100 & 20 \\
    \hline
    \textbf{Sandie's Salads} & 200 & 100 \\
    \hline
  \end{tabular}
  \caption{Stu vs. Sandie}
\end{table}


    What is their cost, in minutes, to produce steak and salads?
\end{frame} 

\begin{frame}{Absolute advantage}
    Assume that there is a constant transferability from one dish to the other:
    \begin{enumerate}
        \item Draw the production possibility frontiers for the two restaurants
        \item Who has the absolute advantage in producing steaks?
        \item Who has the absolute advantage in producing salads?
    \end{enumerate}
\end{frame} 

\begin{frame}{Comparative advantage}
   Before we discuss comparative advantage, let's think about the opportunity cost of each firm for each dish:
   \begin{enumerate}
    \item What are the slopes of the two PPFs?
    \item What is Stu's opportunity cost for producing steaks and salads?
    \item What is Sandie's opportunity cost for producing steaks and salads?
   \end{enumerate}
   In other words: what is the \textit{trade-off} that each restaurant faces as they change their production from one dish to another?
\end{frame} 

\begin{frame}{Comparative advantage}
    The \textit{opportunity cost} of producing salads is the amount of steaks they could have produced with the same input. In our example, this is constant.

    \medskip

    A restaurant has a \textit{comparative advantage} in producing steaks compared to their competitor if their opportunity cost is lower.

    \medskip

    \begin{enumerate}
        \item Can a firm have an absolute advantage in both goods?
        \item Can a firm have a comparative advantage in both goos?
        \item What is the relationship between the comparative advantage in good A and good B?
    \end{enumerate}
\end{frame} 

\begin{frame}{Comparative advantage}
    The comparative advantage in producing good A is the \textit{inverse} of the comparative advantage in producing good B.
    
    \medskip

    If the comparative advantage in good A is high, the comparative advantage for good B must be low.

    \medskip

    Comparative advantage depends on the \textit{opportunity cost}: these concepts are linked.
\end{frame} 

\begin{frame}{Comparative advantage}
    Since most customers like to order a salad with their steak, Sandie and Stu both want to offer both salads and steaks (not necessarily 1-to-1).

    \medskip

    If both spend half their resources on each dish, what is their output?

    \medskip

    Now suppose the two restaurants can trade with each other. What is one set of productions, and one possible trade, which would leave them both better off?

\end{frame} 

\begin{frame}{Comparative advantage}
    When they both split their 1000 minutes 50/50 between the two dishes, their output is:

    \begin{table}[h!]
  \centering
    \begin{tabular}{|c|c|c|}
      \hline
      \textbf{Restaurant} & \textbf{Steaks} & \textbf{Salads} \\
      \hline
      \textbf{Stu's Steakhouse} & 50 & 10 \\
      \hline
      \textbf{Sandie's Salads} & 100 & 50 \\
      \hline
      \textbf{Total output} & 150 & 60 \\
      \hline
    \end{tabular}
    \caption{50/50 split}
    \label{tab:tab3}
  \end{table}



\end{frame} 

\begin{frame}{Possible trade}
    There are many possible answers to this last question, but let's go back to our principle at the beginning of the discussion, and \textit{think at the margin}.
    \begin{itemize}
        \item Stu produces 1 fewer salads and 5 more steaks
        \item Sandie produces 2 fewer steaks, and 1 more salad
    \end{itemize}
    Then their production is:

    \begin{table}[h]
  \centering
    \begin{tabular}{|c|c|c|}
      \hline
      \textbf{Restaurant} & \textbf{Steaks} & \textbf{Salads} \\
      \hline
      \textbf{Stu's Steakhouse} & 55 & 9 \\
      \hline
      \textbf{Sandie's Salads} & 98 & 51 \\
      \hline
      \textbf{Total output} & 153 & 60 \\
      \hline
    \end{tabular}
    \caption{Possible trade}
  \end{table}

    Total production has gone up!
\end{frame} 

\begin{frame}{Possible trade}
    Which trade would leave them both better off?

    \medskip

    Say Stu trades 3 steaks to Sandie in exchange for one salad:

    \begin{table}
    \begin{tabular}{|c|c|c|}
      \hline
      \textbf{Restaurant} & \textbf{Steaks} & \textbf{Salads} \\
      \hline
      \textbf{Stu's Steakhouse} & 52 & 10 \\
      \hline
      \textbf{Sandie's Salads} & 101 & 50 \\
      \hline
    \end{tabular}
    \caption{Gains of trade}
  \end{table}

    They both have the same amount of salads as before, but more steaks! So we can say that they are each better off.

    \medskip
    
    Should they continue to specialize?

\end{frame} 

\begin{frame}{Price of trade}
    Here we just asserted a trade that would make both parties better off in terms of the amount of each dish. But how can we know both parties will agree to the trade?
    
    \medskip

    This is determined by the price of each good. In the example we gave, the ``price'' of one salad was 3 steaks.

    \begin{enumerate}
        \item What if the price of 1 salad was 3.5 steaks?
        \item What if the price of 1 salad was 1 steak?
        \item What if the price of 1 salad was 6 steaks?
    \end{enumerate}
\end{frame}

\begin{frame}{Price of trade}
    The first example would still leave both parties better off, but the second two would not.

    \medskip

    We are not ready yet to discuss where prices come from, but we do have a general rule:

    \begin{center}
        \textit{For trade to make both parties better off, the price must lie between the two opportunity costs.}
    \end{center}

\end{frame}

\begin{frame}{Discussion questions}
    Should Kevin Durant wash his car?

    \medskip

    Should the United States trade with other countries?

    \medskip

    The main takeaway from this chapter:

    \begin{center}
        \textit{Due to comparative advantage, specialization and trade can leave everyone participating better off.}
    \end{center}
\end{frame} 

\begin{frame}{Another example}
    Joseph can peel a pound of potatoes in 10 minutes and wash a load of dishes in 15. Mary can do both of these tasks twice as fast. 

    \medskip

    Which person should do more of which task?
\end{frame}

\begin{frame}{A final example}
    Joseph can peel a pound of potatoes in 10 minutes and wash a load of dishes in 15. Mary can also wash the dishes in 15 minutes, but it takes here only 5 minutes to peel the potatoes. 
    
    \begin{enumerate}
        \item What is each person's opportunity cost of peeling potatoes?
        \item Who has an absolute advantage in washing the dishes?
        \item Who has a comparative advantage in washing the dishes?
        \item If the two workers try and split up the tasks in an advantageous way, who will do more of which job?
    \end{enumerate}

\end{frame}

\begin{frame}{A final example (cont.)}
    Joseph can peel a pound of potatoes in 10 minutes and wash a load of dishes in 15. Mary can also wash the dishes in 15 minutes, but it takes here only 5 minutes to peel the potatoes. 
    
    \medskip

    Think about the price of peeling potatoes in terms of washing dishes. What is the maximum price at which a trade could leave both workers better off? What is the minimum price?

\end{frame}

\begin{frame}
    \frametitle{Textbook}
    Read the textbook, in particular the highlighted vocab and the key concepts! (use Anki or other flashcard app to drill)

    \medskip

    Do extra practice problems
    
    \medskip

    No tricks --- the practice problems from the book will be excellent preparation for the exams, quizes, and homeworks
\end{frame}

\end{document}